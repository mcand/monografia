\usepackage{lmodern}						
\usepackage[T1]{fontenc}		
\usepackage[utf8]{inputenc}		%% Para converter automaticamente acentos como digitados. Mude utf8 para latin1 se precisar. 
                                        %% Permite digitar os acentos no teclado normalmente, sem comandos (\'e \`a , etc.).
\usepackage{lastpage}			
\usepackage{indentfirst}		
\usepackage{color}		
\usepackage[alf,abnt-emphasize=bf,abnt-etal-list=0,abnt-etal-text=it]{abntex2cite}	
\usepackage{graphicx}			
\usepackage{microtype} 	



%% -----------------------------------------------------------------------------

%% Obs.: Alguns acentos foram omitidos.

\titulo{Implementando a metodologia lean no desenvolvimento de software} %%Por exemplo, Titulo da tese
% \subtitulo{: subt\'itulo do trabalho}  %% Retirar o primeiro ``%'' desta linha se for utilizar subtitulo. Deixar os dois pontos antes, em ``: subt\'itulo'' . 
\autor{André Furquim}
\autorR{Sobrenome, Primeiro nome do autor} %%Colocar o sobrenome do autor antes do primeiro nome do autor, separados por ,
\local{Joinville}
\data{2015} %%Alterar o ano se precisar
\orientador[Orientador:]{Maurício Henning} %%Se precisar, troque [Orientador:] por [Orientadora:]
% \coorientador[Coorientador:]{Nome do coorientador } %% Retirar o primeiro ``%'' desta linha se tiver coorientador. Se precisar, troque por [Cooorientadora:]. 
\instituicao{Centro Universitário -- Católica de Santa Catarina}
\faculdade{Instituto de Ci\^encias Exatas} %%Alterar, dentro de chaves {}, se precisar.
\programa{Curso de P\'os\mbox{-Gra}dua\c{c}\~ao em Engenharia de Software} %%Alterar, dentro de chaves {}, se precisar.
\objeto{Disserta\c{c}\~ao (Mestrado)}  %%Tese (Doutorado)
\natureza{Monografia  %%Tese
apresentada ao \insereprograma ~do \insereinstituicao ~como requisito parcial para obten\c{c}\~ao do certificado do curso.} %%Trocar Matem\'atica por outro, se precisar.


%% Abaixo, prencher com os dados da parte final da ficha catalografica

\finalcatalog{1. Palavra-chave. 2. Palavra-chave. 3. Palavra-chave. I. Sobrenome, Nome do orientador, orient. II. T\'itulo.} %% Aqui fica 
% escrito a palavra ``T\'itulo'' mesmo, nao o do trabalho. Se tiver coorientador, os dados ficam depois dos dados 
%% do orientador (II. Sobrenome, Nome do coorientador, coorient.) e antes de ``II. T\'itulo'', o qual passa a ``III. T\'itulo''.

%% ---

\setlength{\parindent}{1.3cm}

\setlength{\parskip}{0.2cm}  

\setlength\afterchapskip{12pt}  
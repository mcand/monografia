% \section{OBJETIVOS}
% \label{chap1:obj}
% O objetivo deste trabalho é propor uma metodologia de aplicação do \textit{lean} em uma empresa de \textit{software} que já utiliza-se de algumas práticas ágeis. Assim, é necessário identificar pontos falhos no modelo de desenvolimento da empresa para melhorar os projetos.
% Para elaborar esse modelo, algumas perguntas precisam ser antes respondidas como:
% \begin{itemize}
% 	\item Como aplicar \textit{lean} no ambiente de desenvolvimento ?
% 	\item Existe uma formula que possa ser aplicada para qualquer empresa ?
% 	\item O quanto as práticas atuais da empresa estão perto ou longe do melhor cenário possível para o \textit{lean} ?
% 	\item Quais ferramentas de \textit{software} podem auxiliar o processo ?
% \end{itemize}
% Ao longo desse trabalho, essas perguntas são respondidas.
% \subsection{Objetivos Específicos}
% Afim de melhorar o ciclo de desenvolvimento, é necessário eliminar os desperdícios (um dos princípios do \textit{lean}). Assim é necessário:
% \begin{itemize}
% \item representar o modelo atual de desenvolvimento de uma maneira que fique bem claro para as pessoas envolvida, tanto a nível de negócio como operacional, os problemas existente; 
% \item Propor um novo modelo, que elimine esses desperdícios e
% \item Sugerir ferramentas que possam melhorar o processo de desenvolvimento.
% \end{itemize}

\chapter{CONCLUSÃO}
\label{chap:conclusao}

Esse trabalho teve como objetivo aplicar métodos de desenvolvimento \textit{lean} em uma empresa de \textit{software}. Através da aplicação de alguns desses princípios, foi notado uma melhora significativa no que tange a manutenção do sistema assim como nos \textit{tickets} recebidos de problemas de suporte relacionado a funcionalidade do teste em questão. 

Para entender e melhorar o processo de desenvolvimento de uma empresa foi importante estudar as diferentes metodologias e suas práticas, como por exemplo o TDD no \textit{Extreme Programming}, vistos no Capítulo \ref{cap:02}. Para se aplicar o \textit{lean} em uma empresa de desenvolvimento não existe uma fórmula correta, pois tudo vai depender da empresa e do negócio. Utilizar-se dos princípios do \textit{lean}, como \textit{kanban} por exemplo, não necessariamente vai ajudar a empresa a eliminar desperdício ou melhorar sua organização. Além de tudo, o desenvolvimento é feito por pessoas e investir na equipe e entender as necessidades da pessoa é fundamental para qualquer projeto.

Para se aplicar \textit{lean} em uma empresa de \textit{software} é necessário entender o desenvolvimento, a equipe e todas as pessoas envolvidas e contrastar com os seus princípios e perguntar-se ``Qual é a melhor maneira de eliminar esse desperdício?''. Práticas como TDD, BDD, \textit{clean code} etc., sem dúvida melhoram o desenvolvimento e qualidade do \textit{software} e vem se tornando um padrão na indústria de desenvolvimento para incorporar qualidade no sistema.


\section{Trabalhos Fufutos}

Os seguinte trabalhos futuros podem ser implementados para a conformidade com o desenvolvimento \textit{lean} e até na área do \textit{lean}:

\begin{itemize}
	\item Implementar integração contínua nos produtos da empresa;
	\item Desenvolver um manual de práticas de codificação com práticas de desenvolvimento para uma empresa;
	\item Aprimorar ainda mais as técnicas de desenvolvimento do \textit{lean} com o desenvolvimento da empresa e
	\item Implementar um modelo para gerenciamento do conhecimento na empresa conforme o princípio do \textit{lean}.
\end{itemize}
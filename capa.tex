%% ELEMENTOS PRE-TEXTUAIS

%% Capa
\inserecapa

%% Folha de rosto
\inserefolhaderosto


%% Ficha catalografica. AO IMPRIMIR, DEIXAR NO VERSO DA FOLHA DE ROSTO.
\inserecatalog  


%% Folha de aprovacao
\begin{folhadeaprovacao}

  \begin{center}
    {\chapterfont \MakeUppercase{\bfseries \insereautor}}

    \vfill
    \begin{center}
      {\chapterfont \MakeUppercase{\bfseries\inseretitulo \inseresubtitulo}}
    \end{center}
    \vfill
    
    \hspace{.45\textwidth}
    \begin{minipage}{.5\textwidth}
        \inserenatureza
        \\ \\
        \begin{center}COMISSÃO EXAMINADORA \end{center}
         \assinatura{Prof. Msc. \insereorientador \ - Orientador \\ Centro Universitário -- Católica de Santa Catarina} 
      %  \assinatura{Professor Dr. \inserecoorientador \ - Coorientador \\ Universidade Federal de Juiz de Fora}
         \assinatura{Prof. }
         \assinatura{Prof. } 
    \end{minipage}%
    \vfill
   \end{center}
           

%  \assinatura{...} %%RETIRE O % E PREENCHA SE PRECISAR
%  \assinatura{...}
%  \assinatura{...}
\end{folhadeaprovacao}


%% Dedicatoria. OPCIONAL. Retirar o ``%'' de cada das 4 linhas abaixo, caso queira.
% \begin{dedicatoria} \vspace*{\fill} \centering \noindent
%   \textit{ Dedico este trabalho ... (opcional)} 
%   \vspace*{\fill}
% \end{dedicatoria}


%% Agradecimentos. OPCIONAL. CASO SEJA BOLSISTA, INSERIR OS DEVIDOS AGRADECIMENTOS.
\begin{agradecimentos}

Este trabalho é decicado à minha família, meus amigos, meu orientador e a Católica de Santa Catarina. 

\end{agradecimentos}

%% Epigrafe. OPCIONAL
% \begin{epigrafe}
%     \vspace*{\fill}
% 	\begin{flushright}
% 		``Texto em que o autor apresenta uma cita\c{c}\~ao, seguida de autoria, relacionada com a                       
%   mat\'eria tratada no corpo do trabalho'' \\
% (ASSOCIA\c{C}\~AO BRASILEIRA DE NORMAS T\'ECNICAS, 2011, p. 2) \\
%   A ep\'igrafe elaborada conforme NBR 10520 (Ep\'igrafe - Opcional)
% 	\end{flushright}
% \end{epigrafe}


%% RESUMOS

%% Resumo em Portugu^es. OBRIGATORIO.
\setlength{\absparsep}{18pt} 
\begin{resumo}
O software como produto foi ganhando cada vez mais importância com o passar do tempo, assim foi necessário investir em modelos que pudessem melhorar seu desenvolvimento para que fosse possível garantir melhor qualidade, custo e prazo no projeto de software. Um modelo que vem ganhando mais notoriedade atualmente é o modelo de desenvolvimento lean, o qual tem seu foco em criar valor para o cliente através da eliminação de desperdícios, otimização do fluxo de valor, capacitação das pessoas e melhora contínua. Esse modelo, criado pela Toyota, embora aplicado em muitas empresas, ainda possui desafios no que tange a sua utilização no mercado de software. Esse trabalho tem como objetivo esclarecer o leitor sobre essa metodologia (além de outras utilizadas no mercado), fazer um comparativo dessas metodologias com o lean, mostrar seus desafios no mercado de software e apresentar maneiras de como proceder para aplicação da metodologia lean de desenvolvimento em uma empresa de software.

\textbf{Palavras-chave}: Lean, Metodologia de Desenvolvimento de Software, Engenharia de Software. %finalizadas por ponto e inicializadas por letra maiuscula.

\end{resumo}
 
 
%% Resumo em Ingle^s
\begin{resumo}[ABSTRACT]
 \begin{otherlanguage*}{english}
 The software as a product has been growing in terms of importance over the time in people’s life, thus it was necessary to invest in models in order to improve software development in the sense of guaranteeing a better quality, cost and time in  a software project. A model that has been growing in importance over the software community is lean, which focuses on creating customer value through waste elimination, optimization of value streams, people empowerment and continuous improvement. Although this model, created by Toyota, is widely applied in several companies, it’s still a challenge regarding to software market. This work aims to explain the reader about this methodology (and other ones used in software market today), compare lean to other software methodologies, show its challenges in the current software market and present ways to implement lean in a software company.

\textbf{Keywords}: Lean, Software Development Methodology, Software Engineering.
 \end{otherlanguage*}
\end{resumo}

%% Seguindo o mesmo modelo acima, pode-se inserir resumos em outras linguas. 

%% Lista de ilustracoes. OPCIONAL.
\pdfbookmark[0]{\listfigurename}{lof}
\listoffigures*
\cleardoublepage


%% Lista de tabelas. OPCIONAL. Retire o ``%'' de cada das 3 linhas seguintes, caso queira.
\pdfbookmark[0]{\listtablename}{lot}
\listoftables*
\cleardoublepage

%% Lista de abreviaturas e siglas. OPCIONAL
\begin{siglas} %%ALTERAR OS EXEMPLOS ABAIXO, CONFORME A NECESSIDADE
  \item[AWS] \textit{Amazon Web Services}
  \item[BDD] \textit{Bevavior Driven Development}
  \item[DSDM] \textit{Dynamic Systems Development Methodology}
  \item[FDD] \textit{Feature Driven Development}
  \item[GQM+] \textit{Goal Question Metric}
  \item[IHC] Interface Humano Computador
  \item[JAD] \textit{Joint Application Design}
  \item[JIT] \textit{Just in Time}
  \item[LSD] \textit{Lean Software Development}
  \item[PSP] \textit{Personal Software Process}
  \item[SAAS] \textit{Software as a Service}
  \item[RUP] \textit{Rational Unified Process}
  \item[TDD] \textit{Test Driven Development}
  \item[TSP] \textit{Team Software Process}
  \item[UAP] \textit{Unified Agile Process} 
  \item[UML] \textit{Unified Modeling Language}
  \item[XP] \textit{Extreme Programming}
\end{siglas}

%% Lista de simbolos. OPCIONAL
% \begin{simbolos} %%ALTERAR OS EXEMPLOS ABAIXO, CONFORME A NECESSIDADE
%   \item[$ \forall $] Para todo
%   \item[$ \in $] Pertence
%  \end{simbolos}

 
%% Sumario
\pdfbookmark[0]{\contentsname}{toc}
\tableofcontents*
\cleardoublepage

%% ----------------------------------------------------------

%% ELEMENTOS TEXTUAIS